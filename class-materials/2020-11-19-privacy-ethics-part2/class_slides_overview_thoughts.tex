\documentclass[aspectratio=169]{beamer}
\setbeamertemplate{navigation symbols}{}
\usepackage{color, amsmath, comment, subfigure}
\usepackage{url}

\usepackage{hyperref}
\hypersetup{
    colorlinks=true,
    linkcolor=blue,
    filecolor=magenta,      
    urlcolor=cyan,
}

%%%%%%%%%%%%%%%%%%%%%%%%%%
\title[]{Thursday, November 19:\\Looking back and looking forward}
\author[]{Matthew J. Salganik}
\institute[]{}
\date[]{COS 597E/SOC 555 Limits to prediction\\Fall 2020, Princeton University}

\begin{document}
%%%%%%%%%%%%%%%%%%%%%%%%%%%
\frame{\titlepage}
%%%%%%%%%%%%%%%%%%%%%%%%%%%
\begin{frame}

Limits to prediction: 
\begin{itemize}
\item Pre-read was full of hypotheses and metrics, but lacking in research designs \pause
\item conditional on data or unconditional on data (or the magic of Imagenet) \pause
\begin{itemize}
\item conditional on data: Fragile Families Challenge \pause
\item unconditional on data: Lorenz, MusicLab, and epidemiology simulations, (Tetlock, maybe) \pause
\item What are problems where limits conditional on data are interesting/important? \pause
\end{itemize}
\item Work that is most exciting to me works from both ends: increasing prediction and measuring limits (e.g., weather) \pause
\item The importance of time (e.g, weather) \pause
\item What can we learn about forecasting (true prediction) from predicting the present (inference)? \pause 
\item What does it mean ``enough data''? \pause
\end{itemize}

\end{frame}
%%%%%%%%%%%%%%%%%%%%%%%%%%%
\begin{frame}

Limits of prediction
\begin{itemize}
\item We rarely care about pure prediction problems. What would it mean to do prediction for intervention? How could we compare the value of improvements to prediction vs improvements to intervention? \pause
\item The time horizon for prediction and action need to be aligned \pause
\item ``Many, many nuances about prediction are domain specific.'' \pause
\item Takes our focus from where it should be: intervention \pause
\end{itemize}

\end{frame}
%%%%%%%%%%%%%%%%%%%%%%%%%%%
\begin{frame}

Predicting the future of prediction
\begin{itemize}
\item Hype cycle
\pause
\item Most the use cases will probably be in inference/automation (e.g., digit recognition)
\pause
\item Also some use cases in helping humans make better predictions (still relatively unexplored)
\pause
\item Principles for designing, testing, deploying, and auditing predictive systems (some social, but some technical, think of google paper on high-interest credit card)
\end{itemize}

\end{frame}
%%%%%%%%%%%%%%%%%%%%%%%%%%%
\begin{frame}

Learning objectives:
\begin{itemize}
\item Students will be able to describe theories of predictability and unpredictability in different scientific domains.
\item Students will be able to compare and correctly apply commonly used measures of predictive performance.
\item Sudents will be able to evaluate the appropriateness of prediction as a scientific or policy goal.
\item Students will be able to make predictions about the future of prediction.
\item Students will be able to create new research that helps understand the limits of predictability.
\end{itemize}

\end{frame}
%%%%%%%%%%%%%%%%%%%%%%%%%%%

\frame{\titlepage}


\end{document}
