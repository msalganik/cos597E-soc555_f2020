\documentclass[aspectratio=169]{beamer}
\setbeamertemplate{navigation symbols}{}
\usepackage{color, amsmath, comment, subfigure}
\usepackage{url}

\usepackage{hyperref}
\hypersetup{
    colorlinks=true,
    linkcolor=blue,
    filecolor=magenta,      
    urlcolor=cyan,
}

%%%%%%%%%%%%%%%%%%%%%%%%%%
\title[]{Comments slides for Tuesday, November 17:\\Privacy and ethics, part 1}
\author[]{Matthew J. Salganik}
\institute[]{}
\date[]{COS 597E/SOC 555 Limits to prediction\\Fall 2020, Princeton University}

\begin{document}
%%%%%%%%%%%%%%%%%%%%%%%%%%%
\frame{\titlepage}
%%%%%%%%%%%%%%%%%%%%%%%%%%%
\begin{frame}

Moral dimension:
\begin{itemize}
\item Sociologists run away (or toward) from moral dimension
\pause
\item Want research to be important, and moral basis is one way to do that
\pause
\item Moral dimension in problem selection, not research design or data analysis (Arvind's point about communicating the findings is a good one for sociologists)
\pause
\item Moral vs adversarial 
\pause
\item Find something that works for you
\end{itemize}

\end{frame}
%%%%%%%%%%%%%%%%%%%%%%%%%%%
\begin{frame}

Kosinki et al.\
\begin{itemize}
\item How to reconcile with the results of the Fragile Families Challenge?
\begin{itemize}
\item digital trace data vs survey data
\pause
\item more cases
\pause
\item ``prediction in the statistical sense'' vs forecasting vs whatever we did in the Fragile Families Challenge
\pause
\item $r$ vs $R^2$. $r = 0.4 \rightarrow R^2 = 0.16$ \pause Why use $r$ anyway?
\end{itemize}
\pause
\item Learning model is simple (linear/logistic regression), data is key
\pause
\item Lots of sparkle (e.g, likely curly fries predicts Raven's Standard Progressive Matrices)
\pause
\item Time stable vs time unstable traits
\pause
\item Does test-retest reliability for say openness create a clear, sharp limit to prediction?
\pause
\item Unanticipated secondary use. \pause Are you worried about that for Fragile Families Challenge? \pause How does Kosinki's other work influence how you think about the ethics of this work?
\end{itemize}

\end{frame}
%%%%%%%%%%%%%%%%%%%%%%%%%%%
\begin{frame}

Duhigg and Christl et al.\
\begin{itemize}
\item Great lead
\pause
\item This seems creepy, but is it unethical? If so, why?
\pause
\item Lack of transparency is a sign of a problem (adding in coupons for lawn mowers and wine glasses to confuse customer) ``And we found out that as long as a pregnant woman thinks she hasn’t been spied on, she’ll use the coupons. She just assumes that everyone else on her block got the same mailer for diapers and cribs. As long as we don’t spook her, it works.''  (Arvind example of FB sharing more information with advertisers than you)
\pause
\item Arvind talked about privacy as a means to power/fairness/justice. Helpful for me. Is this a good strategy?
\pause 
\item Target story part of the hype about machine learning.
\pause
\item Trace data to predict other trace data in the future (compare to trace data to predict survey data) (but why are birth records public?) (is this like nowcasting?)
\pause
\item What if you are doing this stuff to get people to subscribe to the \textit{New York Times}?
\pause
\item What if this was just measuring heterogeneity of treatment effects without inferring sensitive traits?
\pause
\item Can you measure not just tracking but harm?
\end{itemize}

\end{frame}
%%%%%%%%%%%%%%%%%%%%%%%%%%%


\frame{\titlepage}


\end{document}
